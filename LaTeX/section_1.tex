\section{电磁现象的基本规律}
对应电动力学课本第二章,占分20分左右。

\begin{question}
写出电磁现象的基本规律,其中包括真空中的Maxwell方程组,电荷守恒定律,洛仑兹力公式,介质的电磁性质方程及电磁场的边界条件。
\end{question}

\begin{question} 
从毕奥-萨伐尔定律出发推导出磁场的散度。
\end{question}

\begin{question} 从基本的麦克斯韦(Maxwell)方程组出发,推出静电场的基本方程和边界条件。
(需说明静电场的特殊条件,引入电势,将静电场的方程和边界条件用电势来表示)
\end{question}

\begin{question} 证明:通过任意闭合曲面的自由电流和位移电流的总量为零。
\end{question}

\begin{question} 一长直导线半径为 ,电导率为 ,导线中的电流强度为 ,设导线表面带有正电荷 ,计算导体表面外侧的能流密度矢量,并证明单位时间内流入长为 的一段导体的能量,恰好是L长导线的焦耳热损耗。
\end{question}

\begin{question} 叙述法拉第电磁感应定律与楞次定律;由法拉第定律推出变化磁场产生的电场的旋度;麦克斯韦对变化磁场产生的电场的散度做了什么假定?导出电荷产生的电场和变化磁场产生的电场的总电场的散度和旋度。
\end{question}

\begin{question} 1. 从库仑定律可以推出高斯定理,从高斯定理是否可以推出库仑定律?
2. 从毕奥-萨伐尔定律可以推出安培环路定理,安培环路定理是否能代替毕奥-萨伐尔定律?
\end{question}

\begin{question} 推出磁化电流密度与磁化强度之间的关系,并需画图说明。
\end{question}

\begin{question} 推出束缚电荷密度与极化强度之间的关系 ,并需画图以及严谨的说明 。
\end{question}

\begin{question} 静电场,稳定电磁场,电偶极子的电磁场以及电磁波传播的求解问题是普遍的麦克斯韦方程组在特定物理条件下的求解问题。分别写出这些物理条件。
\end{question}

\begin{question} (1)说明什么是电磁感应现象。(2)陈述法拉第电磁感应定律和楞次定律,并结合这两个定律给出统一的数学表达式。(3)指出动生电动势和感生电动势的区别。(4)推出变化磁场引起的电场的旋度。(5)关于变化磁场引起的电场的散度,麦克斯韦做了什么假定。
\end{question}

\begin{question} 从麦克斯韦(Maxwell)方程组的积分形式推出磁场强度 的边界条件。
\end{question}

\begin{question} 已知束缚电荷密度与极化强度的关系为 ,极化电流密度为 ,磁化电流密度与磁化强度的关系为 。从真空中的麦克斯韦方程组出发引入电位移矢量 和磁场强度 推出介质中的麦克斯韦方程组。
\end{question}

\begin{question} 从库仑定律可以推出静电场的散度和旋度,那么从法拉第电磁感应定律能否推出变化磁场引起的电场的散度和旋度?为什么?
\end{question}