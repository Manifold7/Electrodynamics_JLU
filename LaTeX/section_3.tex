\section{稳定电磁场}
对应电动力学课本第四章,占分10分左右。

\begin{question}
一半径为a的均匀带电的导体球壳,绕自身某一直径以角速度$\omega$转动,设球上的总电量为Q(假设始终均匀分布在导体球表面),球内外为真空,计算球内外的矢势和磁场。

(球坐标中方程$\frac{1}{r^2}\frac{\partial }{\partial r}(r^2\frac{\partial f}{\partial r})+\frac{1}{r^2 \sin\theta}\frac{\partial }{\partial \theta}(\sin\theta\frac{\partial f}{\partial \theta})-\frac{f}{r^2\sin^2\theta}=0$的一般解为:$f=\sum_{n=0}^{\infty }(b_nr^n+\frac{a_n}{r^{n+1}})P_n^{(1)}(\cos\theta)$
 ,其中一阶连带勒让德函数$P_n^{(1)}(\cos\theta)=(1-x^2)^{1/2}\frac{\mathrm{d}P_n }{\mathrm{d} x}$)
 \end{question}
 
\begin{question}
一半径为a的均匀磁化介质球,其磁化强度为$\mathbf{M_0}$,求空间静磁荷密度$\rho_m$ ,磁标势及磁场强度。
\end{question}

\begin{question}
半径为a的长直圆柱导体,均匀地沿轴方向通过恒定电流I,导体的磁导率为$\mu_1$ ,周围介质的磁导率为$\mu_2$,求矢势$\mathbf{A}$。
可能用到的公式:柱坐标系中,
对于标量$\phi$有:$\nabla^2\phi=\frac{1}{r}\frac{\partial }{\partial r}(r\frac{\partial \phi}{\partial r})+\frac{1}{r^2 }\frac{\partial^2 \phi}{\partial \theta^2}+\frac{\partial^2\phi}{\partial z^2}$ 

对于矢量$\mathbf{A}$有:$\nabla \times \mathbf{A}=[\frac{1}{r}\frac{\partial A_z}{\partial \theta}]$
\end{question}

\begin{question}
(2007+2011)如图所示, 一无限长的磁导率为  的介质圆柱,半径为 ,置于磁导率为 的磁介质中,在垂直于柱轴方向加一均匀静磁场 ,求柱内、外磁场分布。
	(在柱坐标系中, 的通解为: ,其中 为常数。)
\end{question}

\begin{question}
(2008)如图所示,有一内、外半径分别为 和 的空心球壳,放置在均匀磁场  中,球壳介质的磁导率为 ,球壳外为真空,求空间的磁标势。
\end{question}

\begin{question}
(2009)一半径为 的均匀带电球壳,质量为 ,总电量为 ,绕某一直径以角速度 转动,求磁矩和旋磁比。
\end{question}

\begin{question}
(2010+2013)一半径为 a 的均匀带电球壳 ,总电量为 Q ,绕某一直径以角速度  转动 ,求磁场总能量 。(2013年告诉A)
\end{question}

\begin{question}
(2012)在一长直导线电流I1附近,有一矩形载流线圈,电流为I2,线圈和直导线的最短距离为x,夹角为θ,如图所示,计算线圈所受的力和力矩。

\end{question}